%%%%%%%%%%%%%%%%%%%%%%%%%%%%%%%%%%%%%%%
% Deedy - One Page Two Column Resume
% LaTeX Template
% Version 1.1 (30/4/2014)
%
% Original author:
% Debarghya Das (http://debarghyadas.com)
%
% Original repository:
% https://github.com/deedydas/Deedy-Resume
%
% IMPORTANT: THIS TEMPLATE NEEDS TO BE COMPILED WITH XeLaTeX
%
% This template uses several fonts not included with Windows/Linux by
% default. If you get compilation errors saying a font is missing, find the line
% on which the font is used and either change it to a font included with your
% operating system or comment the line out to use the default font.
%
%%%%%%%%%%%%%%%%%%%%%%%%%%%%%%%%%%%%%%
%
% TODO:
% 1. Integrate biber/bibtex for article citation under publications.
% 2. Figure out a smoother way for the document to flow onto the next page.
% 3. Add styling information for a "Projects/Hacks" section.
% 4. Add location/address information
% 5. Merge OpenFont and MacFonts as a single sty with options.
%
%%%%%%%%%%%%%%%%%%%%%%%%%%%%%%%%%%%%%%
%
% CHANGELOG:
% v1.1:
% 1. Fixed several compilation bugs with \renewcommand
% 2. Got Open-source fonts (Windows/Linux support)
% 3. Added Last Updated
% 4. Move Title styling into .sty
% 5. Commented .sty file.
%
%%%%%%%%%%%%%%%%%%%%%%%%%%%%%%%%%%%%%%%
%
% Known Issues:
% 1. Overflows onto second page if any column's contents are more than the
% vertical limit
% 2. Hacky space on the first bullet point on the second column.
%
%%%%%%%%%%%%%%%%%%%%%%%%%%%%%%%%%%%%%%

\setlength{\parindent}{4em}
\setlength{\parskip}{1em}

\documentclass[]{deedy-resume-openfont}

\usepackage{ragged2e}

\begin{document}

%%%%%%%%%%%%%%%%%%%%%%%%%%%%%%%%%%%%%%
%
%     LAST UPDATED DATE
%
%%%%%%%%%%%%%%%%%%%%%%%%%%%%%%%%%%%%%%
%\lastupdated

%%%%%%%%%%%%%%%%%%%%%%%%%%%%%%%%%%%%%%
%
%     CONTACT INFORMATION
%
%%%%%%%%%%%%%%%%%%%%%%%%%%%%%%%%%%%%%%
%\contact

%%%%%%%%%%%%%%%%%%%%%%%%%%%%%%%%%%%%%%
%
%     TITLE NAME
%
%%%%%%%%%%%%%%%%%%%%%%%%%%%%%%%%%%%%%%
\namesection{Matthieu}{NICOLAS}{ \href{mailto:matthieu.nicolas@inria.fr}{matthieu.nicolas@inria.fr} | 06 75 98 34 40 }

%%%%%%%%%%%%%%%%%%%%%%%%%%%%%%%%%%%%%%
%     EXPERIENCE
%%%%%%%%%%%%%%%%%%%%%%%%%%%%%%%%%%%%%%

\section{Déroulement de carrière}

\runsubsection{Ingénieur Recherche \& Développement}
\descript{| INRIA, equipe COAST}
\location{ CDD | Septembre 2016 – Juin 2017 | Nancy}
\sectionsep

\runsubsection{Ingénieur Jeune Diplômé}
\descript{| INRIA, equipe COAST}
\location{ CDD | Septembre 2014 – Août 2016 | Nancy}
\sectionsep

\runsubsection{Élève-ingénieur}
\descript{| TELECOM Nancy}
\location{Diplôme d'ingénieur TELECOM Nancy, spécialité Ingénierie du Logiciel }
\location{Septembre 2011 – Août 2014 | Nancy}
\sectionsep

\runsubsection{Élève-technicien}
\descript{| IUT de Metz}
\location{DUT Informatique}
\location{Septembre 2009 – Juin 2011 | Metz}
\sectionsep

\section{Projets}

\descript{TVPaint}
TVPaint Animation est un logiciel permettant de réaliser des films d'animation.\\
TODO: Rappeler problématique => passer d'une application desktop à une architecture client/serveur avec composante gestion de projet
\begin{tightemize}
\item Réalisation d'un état de l'art sur les mécanismes d'authentification
\item Modélisation du processus de réalisation d'un film d'animation en utilisant un outil de BPM
\item Conception de la nouvelle architecture de l'application
\item Réalisation d'un POC
\end{tightemize}
\sectionsep

\descript{MUTE}
\href{https://www.coedit.re}{\customboldlink{MUTE}} est un outil d'édition collaborative temps réel développé au sein de l'équipe
pour illustrer ses travaux sur les algorithmes de réplication de données et les mécanismes de maintien de la cohérence à terme.
\begin{tightemize}
\item Mise à jour des technologies utilisées
\item Refonte de l'architecture de l'application
\item Mise en place d'un système d'anti-entropie (TODO: Délivrable OpenPaas)
\item Correction de bugs présents dans l'implémentation de \emph{LogootSplit} utilisée
\end{tightemize}
\sectionsep

\descript{PLM}
\href{http://people.irisa.fr/Martin.Quinson/Teaching/PLM/}{\customboldlink{La PLM}} est un environnement d’apprentissage de la programmation libre et ouvert.
Développé par Gérald Oster et Martin Quinson, il permet d’explorer différents aspects de l’algorithmique au travers d’exercices interactifs et graphiques.
\\
Disponible historiquement sous la forme d'une application lourde Java, le but de ce projet était d'effectuer le portage de cet outil en une application web
afin de le rendre accessible au plus grand nombre.
\\
Ma principale tâche a donc été d'effectuer ce portage.
Ce changement important de type d'application a entraîné l'apparition de plusieurs problématiques auxquelles il a fallu apporter des solutions :
\begin{tightemize}
\item Conception et mise en place d'une architecture distribuée assurant le passage à l'échelle de l'application
\item Isolation de l'exécution du code des apprenants
\item Déploiement et supervision d'une application multi-composants
\end{tightemize}
\sectionsep

\descript{artEoz}
\href{http://arteoz.loria.fr/}{artEoz} est un outil dédié à l'enseignement de la programmation.
Il permet de générer et de visualiser le schéma mémoire correspondant à l'exécution du code Java ou Python saisi par l'utilisateur.\\
Disponible sous la forme d'une application desktop,
\begin{tightemize}
\item TODO
\end{tightemize}
\sectionsep

\runsubsection{INRIA, équipe SCORE}\\
\location{ Stage | Avril 2014 – Août 2014 | Nancy}

\descript{MUTE}
Developpement de \emph{MUTE}, un outil d'edition collaborative temps réel
\begin{tightemize}
\item Implémentation sous forme de librairie de \emph{LogootSplit}, un algorithme de réplication des données et de maintien de la cohérence à terme issu des travaux de l'équipe
\item Implémentation d'un éditeur collaboratif temps réel en ligne reposant sur cette librairie
\end{tightemize}
\sectionsep

\runsubsection{École Polytechnique de Montréal}\\
\location{Stage | Avril 2011 – Juin 2011 | Montréal, Canada}
\descript{Developpement d’un outil d'analyse d'algorithmes d'edition collaborative }
Les outils d'édition collaborative existants reposent majoritairement sur une famille spécifique d'algorithmes
pour assurer le maintien de la cohérence à terme : les transformées opérationnelles.\\
Deux propriétés de convergences \emph{TP1} et \emph{TP2} existent et permettent de prouver la correction de ces algorithmes.\\
L'objectif de ce stage était de réaliser un outil permettant de vérifier automatiquement le respect de ces propriétés pour un algorithme donné.
\begin{tightemize}
\item Implémentation de plusieurs algorithmes issus de la famille des transformées opérationnelles
\item Développement de l'outil permettant de vérifier les propriétés de convergences \emph{TP1} et \emph{TP2} pour les algorithmes implémentés.
\end{tightemize}
\sectionsep

%%%%%%%%%%%%%%%%%%%%%%%%%%%%%%%%%%%%%%
%     TEACHING
%%%%%%%%%%%%%%%%%%%%%%%%%%%%%%%%%%%%%%

\section{Enseignement}

\runsubsection{Bases de la Programmation Orientée Objet}\\
\location{Licence 2 Informatique - Faculté des Sciences et Technologies de Nancy}
TODO: Bases de la POO\\
\begin{tabular}{cp{125mm}}
2017            & Chargé de TP.\newline Préparation et animation de l'atelier du TP noté.
\end{tabular}
\sectionsep

\runsubsection{Préparation informatique}\\
\location{1A ingénieur - TELECOM Nancy}
TODO: bases de l'info grâce à PLM\\
\begin{tabular}{cl}
2015-2016       & Responsable du module\\
2014            & ?
\end{tabular}
\sectionsep

%%%%%%%%%%%%%%%%%%%%%%%%%%%%%%%%%%%%%%
%     MANAGEMENT
%%%%%%%%%%%%%%%%%%%%%%%%%%%%%%%%%%%%%%

\section{Encadrement}

\runsubsection{Simulation du comportement de collaborateurs dans une session d’édition collaborative}
\descript{}
\location{Projet d'initiation à la recherche | Janvier 2017 - Mai 2017}
Co-encadrant de 2 étudiants de 2A ingénieur - TELECOM Nancy
\sectionsep

\runsubsection{Service de compilation isolé pour la PLM}
\descript{}
\location{Stage | Juin 2015 - Août 2015}
Co-encadrant d'un étudiant de 2A ingénieur - TELECOM Nancy
\sectionsep

\runsubsection{Mass Error Mediation in a Learning Environment}
\descript{}
\location{Stage | Juin 2015 - Août 2015}
Co-encadrant d'un étudiant de 2A ingénieur - TELECOM Nancy
\sectionsep

\runsubsection{Mise en place d'un environnement de qualification pour la plateforme PLM}
\descript{}
\location{Stage | Juin 2015 - Août 2015}
Co-encadrant d'un étudiant de 2A ingénieur - TELECOM Nancy
\sectionsep

%%%%%%%%%%%%%%%%%%%%%%%%%%%%%%%%%%%%%%
%     PUBLICATIONS
%%%%%%%%%%%%%%%%%%%%%%%%%%%%%%%%%%%%%%

\section{Publications}
\renewcommand\refname{\vskip -1.5cm} % Couldn't get this working from the .cls file
\bibliographystyle{abbrv}
\bibliography{publications}
\nocite{*}

%%%%%%%%%%%%%%%%%%%%%%%%%%%%%%%%%%%%%%
%     COMMUNICATION
%%%%%%%%%%%%%%%%%%%%%%%%%%%%%%%%%%%%%%

\section{Communication}

\runsubsection{Présentation des travaux autour de MUTE}
\descript{}
\begin{tabular}{cp{150mm}}
13 décembre 2016       & HCERES\\
08 décembre 2016       & Rencontres Inria Industrie "Nouvelles technologies pour la protection des données et des systèmes numériques"\\
01 décembre 2016       & Délégation de Recteurs des universités de technologie du Mexique\\
25 novembre 2016       & Rencontres Inria Industrie "Interaction avec les objets et services numériques"\\
14 octobre 2016        & Évaluation INRIA\\
\end{tabular}
\sectionsep

\runsubsection{Présentation des travaux autour de PLM}
\descript{}
\begin{tabular}{cl}
02 février 2016        & Séminaire Ingénieurs Jeunes Diplômés\\
03 mars 2015           & Séminaire Ingénieurs Jeunes Diplômés\\
\end{tabular}
\sectionsep

%%%%%%%%%%%%%%%%%%%%%%%%%%%%%%%%%%%%%%
%     AWARDS
%%%%%%%%%%%%%%%%%%%%%%%%%%%%%%%%%%%%%%

%\section{Awards}
%\begin{tabular}{rll}
%2014	     & top 52/2500  & KPCB Engineering Fellow\\
%2014	     & 2\textsuperscript{nd} most points  & Google Code Jam, Qualification Round\\
%2014	     & 1\textsuperscript{st}/50  & Microsoft Coding Competition, Cornell\\
%2013	     & National  & Jump Trading Challenge Finalist\\
%2013     & 7\textsuperscript{th}/120 & CS 3410 Cache Race Bot Tournament  \\
%2012     & 2\textsuperscript{nd}/150 & CS 3110 Biannual Intra-Class Bot Tournament \\
%2011     & National & Indian National Mathematics Olympiad (INMO) Finalist \\
%2010     & National & Comp. Soc. of India's National Programming Contest\\
%\end{tabular}
%\sectionsep

%%%%%%%%%%%%%%%%%%%%%%%%%%%%%%%%%%%%%%
%     SOCIETIES
%%%%%%%%%%%%%%%%%%%%%%%%%%%%%%%%%%%%%%

%\section{Societies}

%\begin{tabular}{rll}
%2014 	& top 12\%ile    & Tau Beta Pi Engineering Honor Society\\
%2014   & National   & The Global Leadership and Education Forum (tGELF)\\
%2012   &  National  & Golden Key International Honor Society\\
%2012   &  National   & National Society of Collegiate Scholars\\
%\end{tabular}
%\sectionsep

\hfill
\end{document}  \documentclass[]{article}
